\documentclass{article}
%%%%%%%%%%%%%%%%%%%%%%%%%%%%% Using Packages %%%%%%%%%%%%%%%%%%%%%%%%%%%%%%%%%%
\usepackage{float}
\usepackage{geometry}
\usepackage{graphicx}
\usepackage{amssymb}
\usepackage{amsmath}
\usepackage{amsthm}
\usepackage{empheq}
\usepackage{mdframed}
\usepackage{booktabs}
\usepackage{lipsum}
\usepackage{color}
\usepackage{psfrag}
\usepackage{pgfplots}
\usepackage{bm}
\usepackage{csquotes}
\usepackage[spanish]{babel}
\usepackage[style=apa]{biblatex}
%%%%%%%%%%%%%%%%%%%%%%%%%%%%%%%%%%%%%%%%%%%%%%%%%%%%%%%%%%%%%%%%%%%%%%%%%%%%%%%

%%%%%%%%%%%%%%%%%%%%%%%%%% Page Setting %%%%%%%%%%%%%%%%%%%%%%%%%%%%%%%%%%%%%%%
\geometry{letterpaper, margin=2.54cm}

%%%%%%%%%%%%%%%%%%%%%%%%%% Define some useful colors %%%%%%%%%%%%%%%%%%%%%%%%%%
\definecolor{ocre}{RGB}{243,102,25}
\definecolor{mygray}{RGB}{243,243,244}
\definecolor{deepGreen}{RGB}{26,111,0}
\definecolor{shallowGreen}{RGB}{235,255,255}
\definecolor{deepBlue}{RGB}{61,124,222}
\definecolor{shallowBlue}{RGB}{235,249,255}
%%%%%%%%%%%%%%%%%%%%%%%%%%%%%%%%%%%%%%%%%%%%%%%%%%%%%%%%%%%%%%%%%%%%%%%%%%%%%%%

%%%%%%%%%%%%%%%%%%%%%%%%%% Define an orangebox command %%%%%%%%%%%%%%%%%%%%%%%%
\newcommand\orangebox[1]{\fcolorbox{ocre}{mygray}{\hspace{1em}#1\hspace{1em}}}
%%%%%%%%%%%%%%%%%%%%%%%%%%%%%%%%%%%%%%%%%%%%%%%%%%%%%%%%%%%%%%%%%%%%%%%%%%%%%%%

%%%%%%%%%%%%%%%%%%%%%%%%%%%% English Environments %%%%%%%%%%%%%%%%%%%%%%%%%%%%%
\newtheoremstyle{mytheoremstyle}{3pt}{3pt}{\normalfont}{0cm}{\rmfamily\bfseries}{}{1em}{{\color{black}\thmname{#1}~\thmnumber{#2}}\thmnote{\,--\,#3}}
\newtheoremstyle{myproblemstyle}{3pt}{3pt}{\normalfont}{0cm}{\rmfamily\bfseries}{}{1em}{{\color{black}\thmname{#1}~\thmnumber{#2}}\thmnote{\,--\,#3}}
\theoremstyle{mytheoremstyle}
\newmdtheoremenv[linewidth=1pt,backgroundcolor=shallowGreen,linecolor=deepGreen,leftmargin=0pt,innerleftmargin=20pt,innerrightmargin=20pt]{theorem}{Theorem}[section]
\theoremstyle{mytheoremstyle}
\newmdtheoremenv[linewidth=1pt,backgroundcolor=shallowBlue,linecolor=deepBlue,leftmargin=0pt,innerleftmargin=20pt,innerrightmargin=20pt]{definition}{Definition}[section]
\theoremstyle{myproblemstyle}
\newmdtheoremenv[linecolor=black,leftmargin=0pt,innerleftmargin=10pt,innerrightmargin=10pt]{problem}{Problem}[section]
%%%%%%%%%%%%%%%%%%%%%%%%%%%%%%%%%%%%%%%%%%%%%%%%%%%%%%%%%%%%%%%%%%%%%%%%%%%%%%%

%%%%%%%%%%%%%%%%%%%%%%%%%%%%%%% Plotting Settings %%%%%%%%%%%%%%%%%%%%%%%%%%%%%
\usepgfplotslibrary{colorbrewer}
\pgfplotsset{width=8cm,compat=1.9}
%%%%%%%%%%%%%%%%%%%%%%%%%%%%%%%%%%%%%%%%%%%%%%%%%%%%%%%%%%%%%%%%%%%%%%%%%%%%%%%

%%%%%%%%%%%%%%%%%%%%%%%%%%%%%%% Title & Author %%%%%%%%%%%%%%%%%%%%%%%%%%%%%%%%
\title{Taller Resistencia de Materiales}
\author{Gustavo Vergara}
%%%%%%%%%%%%%%%%%%%%%%%%%%%%%%%%%%%%%%%%%%%%%%%%%%%%%%%%%%%%%%%%%%%%%%%%%%%%%%%

\begin{document}

    \begin{titlepage}
\centering


\vspace{3cm}
{\scshape\Huge GLOBALIZACIÓN Y CULTURA \par}
\vspace{5cm}
\textbf\large\scshape{\par}
     \vspace{4cm}
     
{\Large Vergara Pareja Gustavo\\Vidal Sanchéz Pablo\\Perez Álvarez Mateo\\\par}
\vspace{5cm}
{\scshape\Large HUMANIDADES II \par}
\vspace{0.5cm}
{\scshape\Large Programa de Ingeniería Mecánica \par}
\vspace{1cm}
{\scshape\Large Universidad de Córdoba\par}
\vspace{1cm}
{\Large \today \par}
\end{titlepage}


 \newpage

\section*{Globalización y Cultura}
\addcontentsline{toc}{section}{Globalización y Cultura}

En el artículo se aborda la relación entre la globalización y la cultura, destacando que la cultura está intrínsecamente ligada a todos los aspectos sociales. Para comprender la globalización desde una perspectiva cultural, es necesario considerar su contexto histórico, económico, político y financiero. Desde la conquista de América, la globalización ha estado presente y los aspectos culturales han acompañado a los procesos políticos y económicos. Sin embargo, una diferencia clave en la globalización contemporánea es el cambio tecnológico acelerado.

La globalización cultural implica compartir sistemas de signos, códigos y valores, y se destaca que el consumo avanza sobre la cultura. Es importante considerar la intersección entre lo global y lo local, ya que los consumos y mensajes no son uniformes, sino que son interpretados y asignados significados en el contexto de la cultura local. Se plantea la hipótesis de que existen diferentes niveles de códigos culturales en cada sociedad, desde los particulares hasta los globales, y que están en constante intercambio y transformación. 

En el caso de los países latinoamericanos, incluyendo Argentina, se menciona que han estado incluidos en un sistema mundial de relaciones económicas, políticas y culturales desde el principio. La influencia de la globalización en la consolidación de la identidad cultural de una nación es evidente, ya que la herencia del pasado y las tradiciones coloniales se mezclan con las influencias de la migración y las ideologías del proceso de constitución nacional. Además, la desterritorialización, la influencia de los medios de comunicación y la dominancia de los flujos de información también juegan un papel importante en la formación de la identidad cultural.

Sin embargo, la globalización también plantea desafíos y contradicciones. Por un lado, se menciona que la diversidad cultural se expande debido a los contactos con lo diferente y la abundancia de información. Pero, por otro lado, existe la posibilidad de uniformización de los códigos simbólicos debido a la comunicación sin copresencia.

Además, la globalización cultural también implica la imposición de una cultura dominante, principalmente a través de las empresas transnacionales. Estas empresas tienen un papel fundamental en la difusión de productos culturales y en la imposición de ciertos códigos y valores. A medida que los Estados nacionales pierden poder frente a estas empresas, se produce una homogeneización cultural a nivel global. Esto se ve reflejado en la transformación de los códigos que organizan la percepción del tiempo y el espacio. La comunicación y las transacciones entre diferentes países y regiones requieren la creación de un nuevo ritmo temporal que trascienda las 

diferencias horarias locales. Además, el dinero se convierte en un producto cultural dominante, ya que establece significados compartidos, ritmos y legitimidades en el mundo financiero. El dinero electrónico, que fluye rápidamente a través de las computadoras, se vuelve cada vez más abstracto y desligado de su referente material, convirtiéndose en un símbolo alimentado por la confiabilidad de sus emisores. En este sentido, el dinero se vuelve un producto virtual que circula y se reproduce en el ámbito digital, perdiendo su relación con la riqueza material 

En conclusión, la globalización cultural tiene un impacto significativo en la sociedad contemporánea. Aunque permite la difusión de diferentes culturas y la interacción entre ellas, también plantea desafíos y contradicciones. La imposición de una cultura dominante, la homogeneización cultural y la pérdida de poder de los Estados nacionales son algunos de los aspectos negativos de la globalización. Sin embargo, también es importante reconocer que la globalización cultural puede ser una oportunidad para el intercambio y la diversidad cultural, siempre y cuando se promueva un equilibrio entre lo global y lo local, y se respeten las particularidades de cada sociedad.

    
\end{document}